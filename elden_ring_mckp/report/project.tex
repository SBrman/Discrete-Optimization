\documentclass[10pt]{article}

\usepackage{amsmath}
\usepackage{amsthm}
\usepackage{amsfonts}
\usepackage{hyperref}
\usepackage{cleveref}
\usepackage{array}
\usepackage{float}
\usepackage[round]{natbib}
\usepackage{algorithm}
\usepackage{algpseudocode}
\usepackage{mathtools}

\newtheorem{theorem}{Proposition}
\newtheorem{definition}{Definition}
\newtheorem{lemma}{Lemma}

% \usepackage[top=1in,bottom=1in,left=1in,right=1in]{geometry}

\newcommand{\cbrace}[1]{\left\lbrace #1 \right\rbrace}
\newcommand{\C}{\mathcal{C}}

\newcolumntype{C}[1]{>{\centering\arraybackslash}p{#1}}

\title{
    Multiple Choice Knapsack Problem using Cover Inequalities for Optimizing Armor Selection for Elden Ring.
}
\author{
    \href{barma017@umn.edu}{Simanta Barman}
}

\date{}

\begin{document}

    \begin{titlepage}
        \maketitle
        \thispagestyle{empty}
    \end{titlepage}

    \section{Introduction}
    Elden Ring is an Action Role Playing Game (ARPG) developed by FromSoftware and was released on February 25th, 2022.
    FromSoftware is famous for making very difficult games and they did not make an exception with this game. A player
    is assigned many base stats in the beginning of the game which can be upgraded by making progress in the game. One
    of the most important stats is the equip load. The speed at which the player can move, evade attacks by performing
    dodge rolls and backsteps depends on this stat. Based on the equip load a player can move in four different speeds.
    Equip load under $30\%, 70\%$, greater than or equal to $70\%$ and $100\%$ of the maximum weight of the player puts
    load status of the player to light, medium, heavy and overloaded respectively. Light load allows the player to move
    the fastest and overloaded restricts all evasive actions and allows only very slow movements. Equip loads depends on
    the sum of the weight of the items carried by the player. Items include weapons, armors, talismans and armaments.
    Being the heaviest items and crucial to the protection of the player, armor choice is very important. A player
    has to select four pieces of armor: 1. Head, 2. Arms, 3. Chest, 4. Legs.

    The objective of this project is to find out the best armor pieces from all the armor pieces available in the game
    so that the sum of the weights of the armor pieces is under a maximum weight threshold. The maximum weight or budget
    weight from this point in this report will refer to the maximum sum of armor piece's weight only. A brute force solution 
    implementation is already available at \href{https://github.com/oureuphoriant/Elden_Ring_Armor_Calc}{github}. This
    project will focus on a solution algorithm based on optimization formulation.

    The report is organized in five sections. The first section introduces the objective of the project. The second
    section describes the dataset using which the problem will be tackled. The third section describes the formulation.
    The fourth section is divided into four subsections: 1. 0-1 Knapsack using Dynamic Programming, 2. Separation
    Problem, 3. Lifting Proccedure, 4. Solution Algorithm. The fifth section describes some numerical results using the
    solution methods discussed in the previous sections. Finally the report concludes with the sixth section where some
    conclusions are presented.


    \section{Dataset}
    The \href{https://github.com/oureuphoriant/Elden_Ring_Armor_Calc}{dataset}
    includes the information about the armor pieces. 
    Data for 574 armor pieces are available in the dataset. 
    4 different classes of armor pieces are present in the dataset: 
    \begin{enumerate}
        \item Head piece: $h = 169$ piece's data available.
        \item Arms piece: $a = 94$ piece's data available.
        \item Chest piece: $c = 204$ piece's data available.
        \item Legs piece: $l = 107$ piece's data available.
    \end{enumerate}

    Each armor piece $x$ has some value $p(x) \in \mathbb{R}_+$ for each element $p$ of the set of attributes, $P = \left\lbrace \right.$physical, 
    strike, slash, pierce, magic, fire, lightning, holy, immunity, robustness, focus, poise$\left.\right\rbrace$
    



    \section{Formulation}
    Let, the set of different classes of armors be denoted by $T = \cbrace{H, A, C, L}$. Here the set of all head, arms,
    chest and legs pieces are denoted by $H = \cbrace{H_i}_{i=1}^h$, $A = \cbrace{A_i}_{i=1}^a$, $C =
    \cbrace{C_i}_{i=1}^c$ and $L = \cbrace{L_i}_{i=1}^l$ respectively. Here, $H_i, A_i, C_i$ and $L_i$ are binary
    variables with values $1$ if that piece is chosen in the optimal solution otherwise $0$. Denote the set of all
    attributes for an armor piece by $P$. The weight of item $x \in t$ for all $t \in T$ is denoted by $w(x) \in
    \mathbb{R}_+$. The normalized values for attribute $p \in P$ for an item $x \in t$ for all $t \in T$ can be accessed
    by $p(x)$. Let $m^p$ be the multiplier for each attribute which indicates the priority for that attibute of the
    player. Let $b \in \mathbb{R}_+$ be the maximum weight capacity of the character


    \begin{enumerate}
        \item Choose only $1$ piece of armor from each class of armors.
            \begin{align}
                \sum\limits_{i=1}^h H_i = 1 &&& \text{and, } H_i \in \cbrace{0, 1} \forall i \in \cbrace{1, \cdots, h}\\
                \sum\limits_{i=1}^a A_i = 1 &&& \text{and, } A_i \in \cbrace{0, 1} \forall i \in \cbrace{1, \cdots, a}\\
                \sum\limits_{i=1}^c C_i = 1 &&& \text{and, } C_i \in \cbrace{0, 1} \forall i \in \cbrace{1, \cdots, c}\\
                \sum\limits_{i=1}^l L_i = 1 &&& \text{and, } L_i \in \cbrace{0, 1} \forall i \in \cbrace{1, \cdots, l}
            \end{align}
        \item Weight cannot exceed the maximum weight capacity of the character.
            \begin{align}
                \sum\limits_{i=1}^h w(H_i) \cdot H_i + \sum\limits_{i=1}^a w(A_i) \cdot A_i  
                + \sum\limits_{i=1}^c w(C_i) \cdot C_i  + \sum\limits_{i=1}^l w(L_i) \cdot L_i \leq b
            \end{align}
        \item The objective is to maximize the attibute values based on the priorities set on the attibutes.
            \begin{align}
                \sum\limits_{i=1}^h H_i \left( \sum\limits_{p \in P} p(H_i) \cdot m^p \right) 
                + \sum\limits_{i=1}^a A_i \left( \sum\limits_{p \in P} p(A_i) \cdot m^p \right)  \notag \\
                + \sum\limits_{i=1}^c C_i \left( \sum\limits_{p \in P} p(C_i) \cdot m^p \right)
                + \sum\limits_{i=1}^l L_i \left( \sum\limits_{p \in P} p(L_i) \cdot m^p \right)
            \end{align}
    \end{enumerate}

    Define, the value for piece $x \in t ~\forall t \in T$ by $v(x) = \sum\limits_{p \in P} p(x) \cdot m^p$.

    Then the formulation can be written as the standard Multiple Choice Knapsack Problem (MCKP).

    \begin{align}
        \max \qquad & \sum\limits_{t \in T} \sum\limits_{x \in t} v(x) \cdot x \\
        \text{s.t.} \qquad & \sum\limits_{t \in T} \sum\limits_{x \in t} w(x) \cdot x \leq b &&& \label{kpe1}\\
                            & \sum\limits_{x \in t} x = 1 &&& \forall t \in T \\
                            &  x \in \cbrace{0, 1} &&& \forall t \in T, ~\forall x \in t \label{kpe2}
    \end{align}

    % This problem can be lagrangianized into the following problem by introducing dual variable $u_t > 0$ for all $t \in T$.
    % \begin{align}
    %     \max \qquad & \sum\limits_{t \in T} \sum\limits_{x \in t} v(x) \cdot x - \sum\limits_{t \in T} u_t \left( \sum\limits_{x \in t} x - 1 \right)\\
    %     \text{s.t.} \qquad & \sum\limits_{t \in T} \sum\limits_{x \in t} w(x) \cdot x \leq b &&&\\
    %                         &  x \in \cbrace{0, 1} &&& \forall t \in T, ~\forall x \in t
    % \end{align}

    Now, consider the Knapsack inequality given by \cref{kpe1,kpe2}. The Knapsack set can be written as 
    \begin{align}
        X = \cbrace{ x \in \cbrace{0, 1}^n:  \sum\limits_{t \in T} \sum\limits_{x \in t} w(x) \cdot x \leq b}
    \end{align}

    where, $n = \sum\limits_{t \in T} |t|$ and $N = \cbrace{1,\cdots, n}$. So, an armor piece variable $x_i$ can be
    accessed by index $i \in N$ where $N$ is the set of indeces associated to each piece of armor of different classes.
    The weight function is $w: x \rightarrow \mathbb{R}_+$ and budget for weight is $b \in \mathbb{R}_+$. So, the
    formulation can further be simplified to 
    \begin{align}
        Z_{\mathrm{mckp}} = \max \cbrace{ \sum\limits_{i \in N} v(x_i) \cdot x_i: \sum\limits_{i \in N} w(x_i) \cdot x_i \leq b, \sum\limits_{x \in t} x = 1 ~\forall t \in T, x \in \cbrace{0, 1}^n}
    \end{align}
    

    \section{Methodology}
    We will solve the MCKP by solving the LP relaxation to get a continuous solution. The continuous solution will be
    used to generate cover inequalities by solving the separation problem which will be described later. A lifting
    proccedure will then be used to strengthen the cover inequalities. After adding those inequalities to the LP
    relaxation of the MCKP, the problem will be solved again to get another continuous solution. Then the process will
    be repeated until the separation problem results in an objective value that satisfies all cover inequalities. 

    \begin{definition}[\cite{wolsey} Definition 9.6]
        A set $\C \subseteq N$ is a cover if $\sum_{j \in \C} a_j > b$. A cover is minimal if $\C \backslash \cbrace{j}$ is not a cover for any 
        $j \in \C$.
    \end{definition}

    \begin{theorem}[\cite{wolsey} Proposition 9.2]
        If $\C \subseteq N$ is a cover for $X$, the cover inequality \cref{prop92} is valid for $X$
        \begin{align}
            \sum\limits_{j \in \C} x_j \leq |\C| - 1 \label{prop92}
        \end{align}
    \end{theorem}

    \subsection{0-1 Knapsack with Dynamic Programming}
    The solution proccedure for $0-1$ knapsack with Dynamic Programming (DP) will be used in the following subsections. 
    The DP recursion to solve the 0-1 knapsack is
    \begin{align}
        F[r, h] = \begin{cases}
            F[r-1, h] & \text{ if } a_r^{\mathrm{kp}} > h \\
            \max \cbrace{ F[r-1, h], c_r^{\mathrm{kp}} + F[r-1, h]} & \text{ otherwise}
        \end{cases}
    \end{align}

    Here, the first case means that if the current item $r$ is added to the knapsack then total weight will exceed $h$
    so the current item cannot be included in the knapsack. The second case means that since current item $r$ can be
    included in the knapsack, to maximize to the value of the knapsack choose the case where value is maximized.

    The DP can be solved using the following algorithm given in class:
    \begin{algorithm}
        \caption{Maximum objective value to the 0-1 knapsack using DP}
        \begin{algorithmic}[1]
            \State Inputs: Item values $c^{\mathrm{kp}}$, weights $a^{\mathrm{kp}}$ and budget $b^{\mathrm{kp}}$.
            \State $F[0, h] = 0     \qquad h \in [0, b-1]$
            \State $F[r, 0] = 0     \qquad r \in [1, n-1]$
            \For{$r \in [1, n-1]$}
                \For{$h \in [1, b-1]$}
                    \If {$h \leq a^{\mathrm{kp}}_r - 1$}
                        \State $F[r, h] = F[r-1, h]$
                    \Else
                        \State $\max \cbrace{ F[r-1, h], ~c^{\mathrm{kp}}_r + F[r-1, h-1]}$
                    \EndIf
                \EndFor
            \EndFor
        \end{algorithmic} 
        \label{kpa}
    \end{algorithm}

    Algorithm to get the knapsack solution from the table $F$ created during \cref{kpa}:

    \begin{algorithm}
        \caption{Getting the solution to the 0-1 knapsack using the DP table}
        \begin{algorithmic}[1]
            \State Inputs: $F, a^{\mathrm{kp}}, b^{\mathrm{kp}}$
            \State Initialize solution $x$.
            \State $h \gets b^{\mathrm{kp}} - 1$
            \For{$r \in [n-1, 0]$}
                \If {$F[r, h] = F[r-1, h]$}
                    \State $x_r = 0$
                \Else
                    \State $x_r = 1$
                    \State $h = h - \lfloor a^{\mathrm{kp}}_r \rfloor$
                \EndIf
            \EndFor
        \end{algorithmic} 
    \end{algorithm}

    \subsection{Separation Problem}
    Now the separation problem for the cover inequalities is, given nonintegral $x^*$ with $0 \leq x_j^* \leq 1$ for all $j \in N$ find out 
    whether $x^*$ satisfies all the cover inequalities. \Cref{prop92} can be rewritten as
    \begin{align}
       1 \leq |\C| - \sum\limits_{j \in \C} x_j \notag \\
       \implies 1 \leq \sum\limits_{j \in \C} \left( 1 - x_j \right)
    \end{align}

    The separation problem can be written as a decision problem: Does there exist a set $\C \subseteq N$ with $\sum\limits_{j \in \C} w(x_j) > b$ 
    for which $\sum_{j \in \C} (1 - x_j^*) < 1$? Here, the cover $\C$ would contain all the indeces of the armor piece variables where an index
    indicate an armor piece variable in the flattened list of all armor pieces. This decision problem can be written as an $0-1$ integer program.
    \begin{align} 
        \zeta = \min \quad & \sum\limits_{j \in N} (1 - x_j^*) z_j \\
        \text{s.t.} \quad & \sum\limits_{j \in N} w(x_j) z_j \geq b + 1\\
                    \quad & z_j \in \cbrace{0, 1} && \forall j \in N
    \end{align}

    If $\zeta \geq 1$, $x^*$ satisfies all the cover inequalities (Wolsey Theorem 9.2).
    Here, $z_j = 1$ if $j \in \C$, otherwise $0$. Again, $x_j$ refers to the $j$th armor 
    piece in the flattened list of all armor pieces. 

    So, this knapsack problem can be solved to get the cover inequality for which $x*$ is infeasible. 
    Otherwise if $x^*$ is feasible then cover inequality will not be identified.

    Here, the values, weights, budget for the knapsack instance would be $c^{\mathrm{kp}} = \cbrace{(1 - x_j^*) ~\forall
    j \in N}$, $a^{\mathrm{kp}} = \cbrace{w(x_j) ~\forall j \in N}$, $b^{\mathrm{kp}} = b + 1$. Using the DP described
    in the previous subsection this problem can be solved.


    \subsection{Lifting Proccedure}
    After getting a cover by solving the separation problem with $x^*$ obtained from the LP relaxation of the 
    original MCKP the cover inequalities will be strengthened using a lifting proccedure.

    \begin{enumerate}
        \item Generate different permutations of ordering of $N \setminus \C$
        \item Let, $j_1, \cdots, j_r$ be an ordering of $N \setminus \C$.
        \item Set $t = 1$
        \item The valid inequality $\sum_{i=1}^{t-1} \alpha_{j_i} x_{j_i} + \sum_{j \in \C} x_j \leq |\C| - 1$ 
        has been obtained so far. To calculate largest $\alpha_{j_t}$ for which $\alpha_{j_t} x_{j_t} + \sum_{i=1}^{t-1} \alpha_{j_i} x_{j_i} 
        + \sum_{j \in \C} x_j \leq |\C| - 1$ is valid solve the knapsack problem:
        \begin{align}
            \zeta_t = \max & \sum_{i=1}^{t-1} \alpha_{j_t} x_{j_t} + \sum_{j \in \C} x_j \\
                    \text{s.t.} & \sum_{i \in N \setminus \cbrace{t}} w(x_{j_i}) \cdot x_{j_i} \leq b - w(x_{j_t}) \\
                                & x_i \in \cbrace{0, 1} && \forall i \in N \setminus {t}
        \end{align}
        \item Set $\alpha_{j_t} = |\C| - 1 - \zeta_t$
        \item Increment $t$ by $1$ until $t = r$ and go to step $4$.
        \item Choose a different ordering $j_1, \cdots, j_r$ of $N \setminus \C$ and go to step 3 if the specified maximum number of orderings 
        to test has not been exceeded yet. Otherwise terminate.
    \end{enumerate}

    Since, a lot of differnt permutations of the orderings is possible only a specified number of ordering will be tested. This means that all possible
    cover inequalities may not be identified which may result in not finding the optimal solution to the MCKP.
    
    \subsection{Solution Algorithm}
    Using the separation problem and the lifting proccedure described in the previous section the solution algorithm is given below.
    \begin{enumerate}
        \item Get the LP relaxation of the MCKP.
        \item Solve the LP relaxation of MCKP to get $x^*$.
        \item Using the obtained $x^*$, solve the separation problem to get a cover inequality.
        \item Terminate if the objective value from the separation problem, $\zeta \geq 1$. 
        Because that implies all the cover inequalities are satisfied for $x^*$.
        \item Strengthen the cover inequality using the lifting proccedure.
        \item Add all the inequalities obtained so far to the LP relaxation of MCKP and get new $x^*$.
        \item Go to step 3.
    \end{enumerate}

    \section{Numerical Results}
    The solution algorithms were implemented in Python. The solution algorithms were compared with results from the Gurobi solver.

    For a maximum budget of $b = 30$ the following results were obtained:
    
    \begin{table}[H]
        \centering
        \caption{Performance from different solution methods for maximum budget, $b = 30$.}
        \begin{tabular}{|C{2.8cm}|C{2.5cm}|C{2.5cm}|C{2.5cm}|}
            \hline
            & MCKP with Cover Inequalities (DP) & MCKP with Cover Inequalities (Gurobi) & MCKP (Gurobi)\\
            \hline
            Formulation & Linear Program (LP) & Linear Program (LP) & Integer Program (IP) \\
            \hline
            Time (seconds) & 2.8814997673 & 22.9754681587 & 0.020500421524 \\
            \hline
            Objective Value & 0.176072650791 & 0.176072650791 & 0.180687443257\\
            \hline
            Total Weight & 27.4 & 27.4 & 29.8 \\
            \hline
            Inequalities found & 11 & 8 & --\\
            \hline
        \end{tabular}
        \label{t1}
    \end{table}

    MCKP with cover inequalities was implemented twice. In one of the implementations the separation problema and the
    lifting proccedure was done using the Dynamic Programming (DP) knapsack algorithm. In the other implementation the
    cover and lifting proccedure was done by solving IP's described in the previous sections directly by Gurobi. Gurobi
    took a really long time to solve the separation problem and lifting when IP's were solved. The found solutions using
    the two different implementations did not change. The number of inequalities however changed. The code for both
    implementation will be attached with this report.

    According to \cref{t1}, the time taken to find the knapsack cover inequalities, strengthening them using DP and then
    solving the LP again after adding the inequalities took a lot more time than directly solving the IP with Gurobi.
    The lifting proccedure being time consuming made the MCKP with cover Inequalities much slower than MCKP with Gurobi.
    When the cover and lifting proccedure was done using Gurobi only the run time increased even more. 
    
    Solving the MCKP LP with cover inequalities (both implementations) resulted in a lower objective value than solving
    the MCKP with Gurobi's IP solver. The sum of weight of the armor pieces was also closer to the maximum weight budget
    for MCKP with Gurobi than MCKP with cover inequalities. Because the described lifting proccedure can strengthen
    cover inequalities in different orderings and all possible orderings were not considered because of very large
    number of orderings, more stronger inequalities that exist were not added to the MCKP LP. Because of that MCKP LP
    at the last iteration found $x^*$ which was not the optimal solution and a better $x^*$ would be found if more
    stronger cover inequalities were added.

    A minimum number of iteration is used until which even if $\zeta \geq 1$ from the separation problem the program
    would not terminate. Instead $x^*$ at that point would be slightly modified randomly and another attempt at finding a
    cover will be made. This is done so that more covers can be found with different $x^*$ by solving the separation
    problem more times. 
    
    The cover inequalities \cref{dpci1}--\cref{dpci2} were added to the LP relaxation of the MCKP when DP was used to solve the
    separation problem.
    \begin{gather}
        3x_{338} + 4x_{319} + x_{304} + 5x_{232} + 2x_{224} + x_{38} + x_{30} + x_{28} + x_{26} \notag \\ + x_{23} + x_{20} \leq 7 \label{dpci1} \\
        3x_{338} + 4x_{319} + x_{304} + 5x_{232} + 2x_{224} + x_{38} + x_{30} + x_{28} + x_{26} \notag \\ + x_{23} + x_{20} + x_{15} \leq 7 \\
        3x_{338} + 4x_{319} + x_{304} + 5x_{232} + 2x_{224} + x_{38} + x_{30} + x_{28} + x_{26} \notag \\ + x_{23} + x_{20} + x_{15} + x_{12} \leq 7 \\
        3x_{338} + 4x_{319} + x_{304} + 5x_{232} + 2x_{224} \leq 7 \\
        3x_{338} + 4x_{319} + x_{304} \leq 7 \\
        3x_{338} \leq 7 \\
        3x_{338} + 4x_{319} + x_{304} + 5x_{232} \leq 7 \\
        3x_{338} + 4x_{319} \leq 7 \\
        3x_{338} + 4x_{319} + x_{304} + 5x_{232} + 2x_{224} + x_{38} \leq 7 \\
        3x_{338} + 4x_{319} + x_{304} + 5x_{232} + 2x_{224} + x_{38} + x_{30} \leq 7 \\
        3x_{338} + 4x_{319} + x_{304} + 5x_{232} + 2x_{224} + x_{38} + x_{30} + x_{28} \leq 7 \\
        3x_{338} + 4x_{319} + x_{304} + 5x_{232} + 2x_{224} + x_{38} + x_{30} + x_{28} + x_{26} \leq 7 \\
        3x_{338} + 4x_{319} + x_{304} + 5x_{232} + 2x_{224} + x_{38} + x_{30} + x_{28} + x_{26} + x_{23} \leq 7 \label{dpci2}
    \end{gather}

    The cover inequalities \cref{gci1}--\cref{gci2} were found when Gurobi was used to solve the separation problem:
    \begin{gather}
        x_{544} + x_{519} + x_{500} + x_{414} + x_{342} \leq 3 \label{gci1} \\
        x_{544} + x_{519} + x_{500} \leq 3 \\
        x_{544} \leq 3 \\
        x_{544} + x_{519} + x_{500} + x_{414} \leq 3 \\
        x_{544} + x_{519} \leq 3 \\
        x_{544} + x_{519} + x_{500} + x_{414} + x_{342} + x_{319} \leq 3 \\
        x_{544} + x_{519} + x_{500} + x_{414} + x_{342} + x_{319} + x_{232} \leq 3 \\
        x_{544} + x_{519} + x_{500} + x_{414} + x_{342} + x_{319} + x_{232} + x_{55} \leq 3 \label{gci2}
    \end{gather}

    The selected armor pieces for maximum budget, $b = 30$ is given in \cref{t2}. The MCKP with Cover inequalities using
    DP and Gurobi both resulted in the same solution.
    \begin{table}[H]
        \centering
        \caption{Solution for maximum budget, $b = 30$.}
        \begin{tabular}{|C{1cm}|C{4.8cm}|C{5cm}|}
            \hline
            Class & MCKP with Cover Inequalities & MCKP with Gurobi IP\\
            \hline
            Head & Greathood (weight=5.1) & Greathood (weight=5.1) \\
            \hline
            Arms & Mushroom Arms (weight=1.7) & Mushroom Arms (weight=1.7) \\
            \hline
            Chest & Radahn's Lion Armor (weight=17.5) & Radahn's Lion Armor (weight=17.5) \\
            \hline
            Legs & Mushroom Legs (weight=3.1) & Guardian Greaves (weight=5.5) \\
            \hline
        \end{tabular}
        \label{t2}
    \end{table}
    
    The legs armor piece was not optimal with MCKP with Cover Inequalities because a heavier legs piece could have been
    found like what Gurobi IP found.

    \section{Conclusions}
    In this report the problem of selecting the best armor pieces under a maximum weight budget for the Game Elden Ring
    is considered. First an IP formulation of the problem was given. The LP relaxation of the problem was solved
    repeatedly until separation problem could not identify any violated cover Inequalities. The separation problem was
    also discussed. The cover generated by the separation problem were then strengthened using a lifting proccedure
    which was described as well. After that the entire solution algorithm and the numerical results from the algorithm
    were discussed. Limitations of the solution algorithm were mentioned in the numerical results section. For a
    specified maximum budget the solutions from both methods, MCKP with covers and Gurobi's IP solver, selected the same
    pieces except the leg piece.
    
    Since the problem is not the standard 0-1 Knapsack problem the lifted cover inequalities may not produce the convex hull. So integral solution may not be achieved which is a limitation of the procedure described in this report.

    \bibliographystyle{abbrvnat}
    \bibliography{citations}

    \vspace*{-10pt}

    \section*{Instructions to run code}
    Libraries used:
    \begin{enumerate}
        \item Python (version=``3.9.4'')
        \item Numpy (version=``1.19.5'')
        \item Gurobi (version=``9.1.1'')
    \end{enumerate}

    The ``mckp.py" file can be run directly with Python to see the results.

    \vspace*{-10pt}
    \section*{Data}
    All the data are available in ``data.csv" and \href{https://github.com/oureuphoriant/Elden_Ring_Armor_Calc}{github}.

\end{document}